\documentclass{ctexart}

\usepackage{tikz-14mv}
\usepackage{multicol}
\usepackage{hyperref}
\usepackage{float}
\usepackage{fontspec}
\usepackage{booktabs}


\newfontfamily{\mvfont}{CopperplateCC-Heavy.otf}
\newfontfamily{\engheiti}{SourceHanSansSC-Normal.otf}
\hypersetup
{
    colorlinks = true,
}
\pagestyle{plain}
\title{猴子也能懂的OQ格点定理}
\author{中指君}
\newcommand{\varible}[1]{\engheiti[#1]}

\begin{document}
\maketitle
\section{前言}
OQ格点定理是在\varible{O}\varible{Q}规则下的一个重要而常用的定理, 此定理限制了雷格的可能分布, 简化了很多总雷数相关的推理.
首先, 我们先来回顾一下·\varible{O}\varible{Q}规则的定义:
\begin{itemize}
    \item \varible{O}{\heiti 外部: 非雷区域四方向联通, 雷区域与题板外部四方向联通}
    \item \varible{Q}{\heiti 无方: 每个2x2区域内都至少有一个雷}
\end{itemize}
\section{OQ格点定理}
\begin{multicols}{2}
    在\varible{O}\varible{Q}规则下, 我们可以将雷格分为两类:
\end{document}