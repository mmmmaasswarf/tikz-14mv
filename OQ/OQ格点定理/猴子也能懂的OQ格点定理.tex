\documentclass{ctexart}

\usepackage{amsmath}
\usepackage{amsthm}
\usepackage{amssymb}
\usepackage{tikz}
\usepackage{tikz-14mv}
\usepackage{multicol}
\usepackage{hyperref}
\usepackage{float}
\usepackage{fontspec}
\usepackage{booktabs}
\usepackage{subfig}
\usepackage{todonotes}
\usepackage{xcolor}

\newfontfamily{\mvfont}{CopperplateCC-Heavy.otf}
\newfontfamily{\Noto}{NotoSans-SemiBold.ttf}
\hypersetup
{
    colorlinks = true,
}
\pagestyle{plain}
\title{猴子也能懂的OQ格点定理\ v2.0}
\author{中指君}
\newcommand{\varible}[1]{{\Noto[#1]}}
\newcommand{\variblep}[1]{{[#1]}}
\newcommand{\redcross}[2]{
    \begin{scope}[thick,color = red,shift={(#1,#2)}]
        \draw (0.05,-0.05) -- (-0.05,0.05);
        \draw (-0.05,-0.05) -- (0.05,0.05);
    \end{scope}
}
\newcommand{\yellowcross}[2]{
    \begin{scope}[thick,color = yellow,shift={(#1,#2)}]
        \draw (0.05,-0.05) -- (-0.05,0.05);
        \draw (-0.05,-0.05) -- (0.05,0.05);
    \end{scope}
}
\newcommand{\greencross}[2]{
    \begin{scope}[thick,color = green,shift={(#1,#2)}]
        \draw (0.05,-0.05) -- (-0.05,0.05);
        \draw (-0.05,-0.05) -- (0.05,0.05);
    \end{scope}
}
\newcommand{\greenlinedot}[4]{
    \draw[color = cyan,line width = 1.5] (#1,#2) -- (#3,#4);
    \draw[fill, color = green] (#1,#2) circle (0.03);
    \draw[fill, color = green] (#3,#4) circle (0.03);
    }
\newtheorem{property}{性质}
\newtheorem{theorem}{定理}
\newtheorem{lemma}{引理}
\newtheorem{definition}{定义}

\begin{document}
\maketitle
\tableofcontents
\pagebreak
\section{前言}
OQ格点定理是在\varible{O}\varible{Q}规则下的一系列重要而常用的定理, 此定理限制了雷格的可能分布, 简化了很多总雷数相关的推理.
首先, 我们先来回顾一下\varible{O}\varible{Q}规则的定义:
\begin{itemize}
    \item \varible{O}{\heiti 外部: 非雷区域四方向联通, 雷区域与题板外部四方向联通.}
    \item \varible{Q}{\heiti 无方: 每个$2\times2$区域内都至少有一个雷.}
\end{itemize}

我们讨论主要讨论题板中格点的性质, 现给出格点的定义如下:
\begin{definition}
    题板中横纵线段的交点称为\textbf{格点}.
\end{definition}
需注意的是, 格点并不一定是四个格子的公共点, 题板的边界上也存在格点, 例如: $8\times8$题板含有$(8+1)^2 = 81$个格点.

另外\varible{O}\varible{Q}题板有三个重要性质, 在OQ格点定理的证明中有着重要的意义, 现叙述如下:

\begin{property}
    对于任意$2\times2$以上题板, 其存在最小雷数且不为$0$.
\end{property}
\begin{proof}
    由\varible{Q}规则可知, 题板的每个$2\times2$区域内至少有一个雷, 且雷的分布情况是有限的, 因此题板存在最小雷数且不为$0$.
\end{proof}
\begin{property}
    \label{p2}
    非雷区域的边界为简单多边形.
\end{property}
该结论是直观的, 但其证明稍显复杂, 见附录\ \ref{proof:SimplePolygon}
\begin{property}
    \label{p3}
    非雷区域内部没有格点.
\end{property}
\begin{proof}
    假设非雷区域内部有格点, 则以该格点为中心的$2\times2$区域内无雷, 这与\varible{Q}规则矛盾, 因此, 非雷区域内部没有格点.
\end{proof}
为了方便以后的讨论,我们将格点按其与雷的相对位置的不同分为两类: 
\begin{definition}
    题板中位于非雷区域的边界的格点为\textbf{边界格点}.
\end{definition}
\begin{definition}
    题板中除边界格点以外的格点为\textbf{雷格点}.
\end{definition}
如图\ \ref{gridpoint}\ 所示, 非雷区域由蓝线框出, 边界格点由绿色圆点标记, 雷格点由红叉标记.
\begin{figure}[ht]
    \centering
    \begin{tikzpicture}
        \begin{fmv}{4}{4}
            f & f &   &   \\
              & f & f &   \\
              & f & f &   \\
              &   &   &
        \end{fmv}
        \redcross{0}{0}
        \redcross{0.5}{0}
        \redcross{1}{-1}
        \greenlinedot{1}{0}{1.5}{0}
        \greenlinedot{1.5}{0}{2}{0}
        \greenlinedot{2}{0}{2}{-0.5}
        \greenlinedot{2}{-0.5}{2}{-1}
        \greenlinedot{2}{-1}{2}{-1.5}
        \greenlinedot{2}{-1.5}{2}{-2}
        \greenlinedot{2}{-2}{1.5}{-2}
        \greenlinedot{1}{-2}{1.5}{-2}
        \greenlinedot{1}{-2}{0.5}{-2}
        \greenlinedot{0}{-2}{0.5}{-2}
        \greenlinedot{0}{-2}{0}{-1.5}
        \greenlinedot{0}{-1}{0}{-1.5}
        \greenlinedot{0}{-0.5}{0}{-1}
        \greenlinedot{0}{-0.5}{0.5}{-0.5}
        \greenlinedot{0.5}{-1}{0.5}{-0.5}
        \greenlinedot{0.5}{-1}{0.5}{-1.5}
        \greenlinedot{0.5}{-1.5}{1}{-1.5}
        \greenlinedot{1.5}{-1.5}{1}{-1.5}
        \greenlinedot{1.5}{-1.5}{1.5}{-1}
        \greenlinedot{1.5}{-0.5}{1.5}{-1}
        \greenlinedot{1.5}{-0.5}{1}{-0.5}
        \greenlinedot{1}{0}{1}{-0.5}
    \end{tikzpicture}
    \caption{格点的分类}
    \label{gridpoint}
\end{figure}
\section{OQ格点定理}
\subsection{Pick定理}
在具体证明OQ格点定理之前, 我们需要先介绍一个重要的定理, 作为OQ格点定理证明的核心:
\begin{lemma}[Pick定理]
    \label{pick}
    对于一个简单多边形, $I$为多边形内部的格点数, $B$为多边形边界的格点数, 则多边形的面积$A$满足:
    \begin{equation}
        A = I + \frac{B}{2} - 1.
    \end{equation}
\end{lemma}
这是一个非常经典的定理, 其证明略.

\subsection{\varible{O}\varible{Q}题板的最小雷数}
为了便于理解, 我们先从研究\varible{O}\varible{Q}题板的最小雷数开始, 这是OQ格点定理的一个特殊情况.

对于一个$m\times n$的题板, 设题板的雷数为$R$, 则非雷区域面积为$mn-R$, 总格点数为$(m+1)(n+1)$, 根据\hyperref[p2]{性质2}, 非雷区域内部格点数为0, 对非雷区域应用\ \hyperref[pick]{Pick定理}, 则有:
\begin{equation}
    \label{basic equation}
    mn - R = \frac{B}{2} - 1.
\end{equation}
由此, 我们可以得到\varible{O}\varible{Q}题板的总雷数满足:

\begin{align}
    R &= mn - \frac{B}{2} + 1 \nonumber\\ 
      &\leqslant mn - \left\lceil \frac{(m+1)(n+1)}{2} \right\rceil + 1 \nonumber\\
      &= \left\lceil \frac{(m-1)(n-1)}{2} \right\rceil.
\end{align}
对于这个结论, 要做两点说明:
\begin{enumerate}
    \item 这里只证明了$R$的下界为$\left\lceil \frac{(m-1)(n-1)}{2} \right\rceil$, 不仅如此, 实际上这个数还是下确界, 关于其为下确界的证明, 参见附录\ \ref{inf}.
    \item 由于其中所有的变量都为整数, 所以引入了向上取整符号, 这同时也提醒我们, 在\varible{O}\varible{Q}题板中, 非雷区域的边界格点的数目必为偶数, 换言之, 雷格点数目也受到限制.
    
    为了便于理解, 我们在这里举一个例子: 使用上式计算可得: $2\times4$题板的最小雷数为$2$, 其分布仅有如图\ \ref{example}\ 上的三种情况(仅经过旋转和反射变换可以互相转化的, 认为是同种情况, 下同), 由于总格点数为奇数, 而边界格点数目为偶数, 所以雷格点数目为奇数且为1, 可直观地用图形验证(雷格点用红叉标出).

    事实上,雷格点的分布就是OQ格点定理要讨论的内容.
\end{enumerate}
\begin{figure}[ht]
    \centering
    \begin{tikzpicture}
        \begin{fmv}{2}{4}
        \\
        & f & f \\
        \end{fmv}
        \redcross{1}{-1}
    \end{tikzpicture}
    \quad
    \begin{tikzpicture}
        \begin{fmv}{2}{4}
        \\
        f & & f \\
        \end{fmv}
        \redcross{0}{-1}
    \end{tikzpicture}
    \quad
    \begin{tikzpicture}
        \begin{fmv}{2}{4}
        f \\
        & & f & \\
        \end{fmv}
        \redcross{0}{0}
    \end{tikzpicture}
    \caption{$2\times4$题板的例子}
    \label{example}
\end{figure}

\subsection{OQ格点定理的一般形式}
有了前面的铺垫, 我们几乎就可以得到OQ格点定理了, 有兴趣的读者可以尝试利用以上内容得到自己的结论(提示:考虑总雷数$R$为任意的情况).

接下来给出较为常用的一个结论:
\begin{theorem}[OQ格点第一定理]
    对于总雷数为$R$的$m\times n\ $\varible{O}\varible{Q}题板, 雷格点数M满足:
    \begin{equation}
        M = 2R - (m-1)(n-1).
    \end{equation}
    \label{OQ1}
\end{theorem}
\begin{proof}
    由式\ \ref{basic equation}\ 可知, 
    $$
    B = 2(mn - R -1),
    $$
    则雷格点数为
    \begin{align*}
        M &= (m+1)(n+1) - B \\
          &= (mn + m + n +1) -2(mn - R -1) \\
          &= 2R - (m-1)(n-1).
    \end{align*}
\end{proof}

在实际游玩中, 题板大小和总雷数是固定的, 也即:
\begin{itemize}
    \item 对于$5\times5$题板, $R = 10, M = 2R - (5-1)^2 = 4$;
    \item 对于$6\times6$题板, $R = 14, M = 2R - (6-1)^2 = 3$;
    \item 对于$7\times7$题板, $R = 20, M = 2R - (7-1)^2 = 4$;
    \item 对于$8\times8$题板, $R = 26, M = 2R - (8-1)^2 = 3$.
\end{itemize}

\subsection{OQ格点定理的推论}
在实际游玩过程中, 还有一些情况, 仅仅使用OQ格点第一定理仍然难以得出结论, 这时我们可以使用OQ格点定理的增强形式. 不过, 这一推论的证明更加复杂, 所以我们仍先从一个例子开始慢慢讲解.

我们仍然从$2\times4$题板开始, 当$R = 2$时, 如前, 雷格点数为1. 让我们进一步观察雷格点的分布, 可以发现, 雷格点的位置只可能出现在如图\ \ref{example2}\ 所示的六个位置之一(用黄叉标记).
\begin{figure}[ht]
    \centering
    \begin{tikzpicture}
        \begin{fmv}{2}{4}
        \end{fmv}
        \yellowcross{0}{0}
        \yellowcross{0}{-1}
        \yellowcross{1}{0}
        \yellowcross{1}{-1}
        \yellowcross{2}{0}
        \yellowcross{2}{-1}
    \end{tikzpicture}
    \caption{雷格点的可能位置}
    \label{example2}
\end{figure}

再看$R = 3$的情况, 这时雷格点数为3, 共有4种情况, 如图\ \ref{example3}\ 所示.
\begin{figure}[ht]
    \begin{tikzpicture}
        \begin{fmv}{2}{4}
            f & f & f 
        \end{fmv}
        \redcross{0}{0}
        \redcross{1}{0}
        \greencross{0.5}{0}
    \end{tikzpicture}
    \begin{tikzpicture}
        \begin{fmv}{2}{4}
            f & f \\
              &   &   & f \\
        \end{fmv}
        \redcross{0}{0}
        \redcross{2}{-1}
        \greencross{0.5}{0}
    \end{tikzpicture}
    \begin{tikzpicture}
        \begin{fmv}{2}{4}
            f & & f\\
            f \\
        \end{fmv}
        \redcross{0}{0}
        \redcross{0}{-1}
        \greencross{0}{-0.5}
    \end{tikzpicture}
    \begin{tikzpicture}
        \begin{fmv}{2}{4}
            f & & f & f \\
        \end{fmv}
        \redcross{0}{0}
        \redcross{2}{0}
        \greencross{1.5}{0}
    \end{tikzpicture}
    \caption{$R=3$时雷格点的分布}
    \label{example3}
\end{figure}

你可能注意到, 这里分别使用了红色和绿色的叉来标记雷格点, 标注规则是这样的: 首先, 将网格看成一个平面直角坐标系, 以左上角为原点, 每个格子的长宽均为1, 横纵坐标之和为偶数的格点定义为\textbf{偶格点}, 横纵坐标之和为奇数的格点定义为\textbf{奇格点}. 若雷格点为偶格点, 则用红色叉标记, 若为奇格点则用绿色叉标记.

图\ \ref{example2}\ 中可以看到, 三个雷格点中有两个偶格点, 一个奇格点. 对于前述的$R = 2$的情况, 雷格点数为1 ,一定是偶格点. 其一般规律为:
\begin{theorem}[OQ格点第二定理]
    对于任意\varible{O}\varible{Q}题板中的雷格点, 其中偶格点数为$E$, 奇格点数为$O$, 则有:
    \begin{equation}
        E - O = 0\ \text{或}\ 1.
    \end{equation}    
\end{theorem}
\begin{proof}
    若题板上不全为雷, 由\hyperref[p2]{性质2}\ 可知, 非雷区域的边界为简单多边形, 任取一个格点作为起点, 沿着边界格点的顺序遍历,直到回到起点, 可保证除起点外每个点会且只会经过一次. 在遍历过程中, 偶格点和奇格点交替出现, 因此边界格点中奇偶格点数目相等. 特殊地, 若题板上全为雷, 边界格点数为0, 同样满足以上结论. 综上所述, 边界格点中奇偶格点数目相等.

    因此, 若$m,n$均为偶数, 则题板中偶格点比奇格点多1, 除去其中的边界格点, 雷格点中偶格点数比奇格点数多1; 反之同理, 雷格点中偶格点数与奇格点数相等, 综上所述, 雷格点中偶格点数比奇格点数多0或1.
\end{proof}

在实际游玩中, 进一步有:
\begin{itemize}
    \item 对于$5\times5$题板, 偶雷格点数为2, 奇雷格点数为2;
    \item 对于$6\times6$题板, 偶雷格点数为2, 奇雷格点数为1;
    \item 对于$7\times7$题板, 偶雷格点数为2, 奇雷格点数为2;
    \item 对于$8\times8$题板, 偶雷格点数为2, 奇雷格点数为1.
\end{itemize}

\section{其他规则下的格点定理}

\begin{flushleft}
    \textbf{注意}: \emph{本节旨在讨论其他规则下具有类似思路的格点定理, 并非正文部分, 因此可能会舍弃部分严谨性. 另外, 本节讨论的范围包含非原版游戏内容, 适合对游戏规则有一定了解的读者阅读.}
\end{flushleft}

在讨论其他规则的格点定理之前, 我们先来回顾一下OQ格点定理的证明过程, 很容易就能发现 Pick 定理在证明中起到了核心作用. 大家是否有这样的疑问: 为什么这个普遍成立的定理在除了\varible{O}\varible{Q}以外的题目中难以直接使用呢? 换言之, \varible{O}\varible{Q} 有什么特殊性呢? 现在可以停下来思考一下这个问题, 我会在本节稍后给出我的理解.

\subsection{格点类型的扩充}
在此之前, 让我们先为接下来的讨论做一些准备工作. 为了使格点定理能够适用于更普遍的场景, 我们需要对格点种类进行更细致的划分, 具体而言, 将格点按其处在题板的位置和其周围格子的状态分为7类\footnote{我知道从2到7是一个巨大的跨越, 但是不要感到恐慌, 你马上就会感受到这是自然而且必要的.}.
分类方式如下所述.

首先根据格点处在题板的位置, 将格点分为\textbf{内部格点}和\textbf{边角格点}. 其中边角格点是指位于题板边界的格点, 其周围有1或2个格子; 内部格点是指不位于题板边界的格点, 其周围有4个格子. 用加一撇$(^\prime)$的大写字母表示边角格点, 不加撇的大写字母表示内部格点.

然后, 根据格点周围格子的状态, 将格点分为\textbf{雷格点}, \textbf{边界格点}, \textbf{数字格点}和\textbf{扭断格点}, 其中:
\begin{itemize}
    \item \textbf{雷格点}: 格点周围的全部格子均为雷的格点, 用$M$表示;
    \item \textbf{数字格点}: 格点周围的全部格子均为非雷的格点, 用$N$表示;
    \item \textbf{扭断格点}: 格点周围的4个格子为``扭断''结构, 即
        \begin{tikzpicture}
            \begin{fmv}[
                tmargin = 1pt,
                lmargin = 1pt,
                rmargin = 1pt,
                bmargin = 1pt,
                scale = 1]{2}{2}
            ? & f \\
            f & ?
            \end{fmv}
        \end{tikzpicture}
        或
        \begin{tikzpicture}
            \begin{fmv}[
                tmargin = 1pt,
                lmargin = 1pt,
                rmargin = 1pt,
                bmargin = 1pt,
                scale = 1]{2}{2}
            f & ? \\
            ? & f
            \end{fmv}
        \end{tikzpicture}
        的格点, 用$X$表示;
    \item \textbf{边界格点}: 格点周围的中, 既有雷也有非雷, 但不为扭断结构的格点, 用$B$表示.
\end{itemize}
综合这两种分类方法, 即得到共7种格点类型, 为了方便以后的讨论, 我给他们分别赋予了字母记号. 如表\ \ref{gridpoint types}\ 所示, 需要注意其中的一些记号的意义相较前文已经有所改变.
\begin{table}
    \caption{格点的分类和其对应记号}
    \label{gridpoint types}
    $$
    \text{格点}
    \begin{cases}
        \text{内部格点}
        \begin{cases}
            \text{内部雷格点}\ &M\\
            \text{内部边界格点}\ &B\\
            \text{内部数字格点}\ &N\\
            \text{扭断格点}\ &X
        \end{cases}\\
        \text{边角格点}
        \begin{cases}
            \text{边角雷格点}\ &M'\\
            \text{边角边界格点}\ &B'\\
            \text{边角数字格点}\ &N'
        \end{cases}
    \end{cases}
    $$
\end{table}

有了这么多种格点种类, 我们可以很快给出几个重要恒等式: 对于一个$m\times n$的题板, 其总格点数为$(m+1)(n+1)$, 则有:
\begin{equation}
    \label{total grid points}
    (m+1)(n+1) = M + B + N + X +M' + B' + N'.
\end{equation}
这是显然的, 格点总数等于各类格点数之和.
另外, 如果只考虑内部格点, 则有:
\begin{equation}
    \label{internal grid points}
    (m-1)(n-1) = N + B + N + X.
\end{equation}
两式相减, 即得边界格点对应的方程:
\begin{equation}
    \label{boundary grid points}
    2(m+n) = B' + M' + N'.
\end{equation}

对格点类型进行扩充后, 我们终于可以在格点的视角下重新审视\varible{O}\varible{Q}规则了. \varible{O} 和\varible{Q} 同属于左线规则, 要求雷和非雷的分布满足一定的条件. 规则\varible{O}保证了非雷区域只有一个, 且不存在扭断, 这给我们使用 Pick定理提供了可能, 得以将格数和格点数联系起来. 而\varible{Q} 规则则保证了非雷区域不存在内部格点, 也就是\textbf{内部数字格点} 为0, 记作$N = 0$. 因此我们可以对非雷区域应用 Pick定理, 得到:
\begin{equation}
    \label{pick in OQ}
    mn - R = \frac{B + B' + N'}{2} - 1.
\end{equation}
因此\varible{O}\varible{Q}在格点视角下可总结为:
$$
\begin{cases}
    N = 0,\\
    B + B' + N' = 2(mn - R + 1),\\
    X = 0.
\end{cases}
$$
代入式\ \ref{total grid points}\ 即可得到.
\begin{equation}
    (m+1)(n+1) = M + 2(mn - R + 1) + M'.
\end{equation}
即
\begin{equation}
    M + M' = 2R - (m-1)(n-1).
\end{equation}
这就是\ \hyperref[OQ1]{OQ格点第一定理}.(再次提醒, 这里的$M$和$M'$与前文中使用的记号意义不同.)

实际上OQ格点定理之所以成立, 就是因为他给出了足够多的格点的数量特征, 足以让我们消去式\ \ref{total grid points}\ 中足够多的量, 使我们可以将目光着眼于少数几个量(在这里是雷格点数)满足的数量关系, 这同样也是格点定理的核心.

\subsection{QS格点定理}

其他规则也可以转化为其他的格点方程, 从而构成其他格点定理. 让我们先着眼于 QS格点定理.
\begin{flushleft}
    \varible{S} {\heiti 蛇: 所有雷构成一条蛇. 蛇是一条宽度为1的四联通路径, 不存在分叉、环、交叉.}
\end{flushleft}
显然在\varible{S}中不存在内部雷格点, 即$M = 0$; 另外, \varible{S} 中的雷区可能包含孔洞, 不能直接使用 Pick定理, 不过其满足类似的形式, 如下.
\begin{theorem}[蛇面积方程]
    蛇的面积可以由以下式子描述:
    \begin{equation}
        R = \frac{B + M' + B'}{2} + X - 1.
    \end{equation}
\end{theorem}
此定理的证明见\ \ref{proof:snack-area}.
因此\varible{S}对应的格点方程为
$$
\begin{cases}
    M = 0,\\
    B + 2X + M' + B' = 2R + 2.
\end{cases}
$$
结合式\ \ref{total grid points}\ 可得
$$
N + N' - X = (m+1)(n+1) - 2R - 2.
$$
如果你愿意, 可以把上式叫做S格点定理, 但是显然这个定理不太实用, 因为$N$这个量在实际游玩中很难确定.
不过如果加上规则\varible{Q}, 也就是$N = 0$, 事情就好办多了, 这时:
\begin{equation}
    \label{QS1}
    N' - X= (m+1)(n+1) - 2R - 2.
\end{equation}
不过这个式子仍然不够实用, 因为在实际题板中 $N'$ 的数值较大, 可以用式\ \ref{boundary grid points}\ 消去$N'$, 得到:
\begin{theorem}[QS格点定理]
    对于\varible{Q}\varible{S}题板, 有
    \begin{equation}
       X + B' + M' =2R + 2 - (m-1)(n-1).
    \end{equation}
\end{theorem}
实际游玩中
$$
X + B' + M' = \begin{cases}
    6, \ \text{对于}\ 5\times5\ \text{和}\ 7\times7\ \text{题板},\\
    5, \ \text{对于}\ 6\times6\ \text{和}\ 8\times8\ \text{题板}.
\end{cases}
$$
观察式\ \ref{QS1}\ 我们可以发现, 这几乎就是OQ格点定理的反面, 仅仅就是引入了一个在\varible{S} 中存在的扭断, 然后把所谓的雷格点换成了数字格点而已. 这是很自然的, 因为\varible{O}的性质和\varible{S}十分相似, 它们对应的格点方程的结构也是类似的.

\subsection{更多的格点定理}
在明白了上文的思路之后, 我们可以批量``生产''出其他组合的格点定理, 试将其他左线规则的格点方程列如下:

\begin{itemize}
    \item \varible{O}{\heiti 外部: 非雷区域四方向联通, 雷区域与题板外部四方向联通.}
        $$
        \begin{cases}
            B + B' + N' = 2(mn - R + 1),\\
            X = 0.
        \end{cases}
        $$
    \item \varible{Q}{\heiti 无方: 每个$2\times2$区域内都至少有一个雷.}
        $$
        N = 0
        $$
    \item \varible{D}{\heiti 雷均由$1\times2$或$2\times1$的矩形组成.}
    
    单独看每个矩形对应6个格点, 矩形的角邻接时产生扭断, 因此雷周边总格点数为$6R/2$(扭断应算两次).
        $$
        \begin{cases}
            M = 0,\\
            3R = 2X + B + B' + M'
        \end{cases}
        $$
    \item \varible{S} {\heiti 蛇: 所有雷构成一条蛇. 蛇是一条宽度为1的四联通路径, 不存在分叉、环、交叉.}
        $$
        \begin{cases}
            M = 0,\\
            B + 2X + M' + B' = 2R + 2.
        \end{cases}
        $$
    \item \varible{D'} {\heiti 战舰: 每个雷区域为宽度为1、长度不超过4的矩形, 矩形不能对角相邻}
    
    对于每个矩形, 4个格点位于角上, 边上的格点数目为$[2(\text{矩形尺寸}-1)]$, 对所有矩形求和, 共有$(4\times\text{矩形数目})$个角, $[2(R-\text{矩形数目}-1)]$个边, 因此格点方程为:
            $$
            \begin{cases}
                M = 0,\\
                X = 0,\\
                B' + B + M' = 2R + 2\times\text{矩形数目}.
            \end{cases}
            $$
    如果我们愿意将内部边界格点进一步细分的话, 我们就可以更好区分出每个矩形角上的格点和边上的格点, 以写出更加详细的方程.
    \item \varible{H} {\heiti 横向: 所有雷不与其他雷横向相邻.}
        $$
        M = 0
        $$
    \item \varible{U} {\heiti 一元: 所有雷不与其他雷相邻}
    类似于\varible{D}
        $$
        \begin{cases}
            M = 0,\\
            4R = 2X + B + B' + M'.
        \end{cases}
        $$
        以上两个规则限制性很强, 可以断定其构成的格点定理有效性不强.
    \item \varible{2G} {\heiti 四连块: 所有四联通雷区的面积为4}
        对单个四连块
        $$
        4 = \frac{B + B' + M'}{2} + M - 1.
        $$
        考虑所有四连块时, 扭断应算两次. 因此有:
        $$
        B + B' + M' + 2X + 2M = \frac{5}{2}R 
        $$
        其中$M$的数量较小
    % \item \varible{2F} {\heiti 花田: 染色格中的雷周围4格内恰好有1个雷(默认使用棋盘格染色法)}
    \item \varible{2G'} {\heiti 三连块: 所有四联通雷区的面积为3}
    类似地
    $$
    \begin{cases}
        M = 0,\\
        B + B' + M' + 2X = \dfrac{8}{3}R
    \end{cases}
    $$
\end{itemize}

列出了这么多式子, 很容易发现, 仅有\varible{Q}一个规则涉及到了对$B$的限制, 这足以说明\varible{Q} 在格点定理中的核心地位. 我们几乎可以说, 有了\varible{Q}才有了各式各样的格点定理! 现在, 写出属于你的格点定理吧!

最后, 我们曾经对 OQ格点定理中格点的奇偶性进行了讨论, 实际上其他的格点定理, 也可以将其推广到带奇偶性的情况.
对于\varible{S}, 我们可以继续沿用之前的思路. 但是如果考虑那些存在多个区域的规则时, 就需要做出一点改变, 不过这并不是一件难事. 这部分内容就留给感兴趣的读者自行进行吧.


\pagebreak

\section{致谢与感言}
\subsection{第二版感言}
本人在今年五一完成了本文的第一版, 留下了会介绍其他规则下的格点定理的大饼, 然后本文就被彻底搁置了. 然而群友Jsky单推人在前几天突然来加我好友, 询问我格点定理的进度, 这时我才知道我之前一时兴起写的这篇文章竟然还被人关注着. 这之后我很快投身了对此的研究, 最终将成果整理成本文的第三部分.

尽管最近玩扫雷的时间和频率都明显减小, 但是当我重新拾起这项工作时, 我仍然感到兴奋和享受, 并且投入了很大的热情. 正因如此我仅用了一个周末的时间就完成了大部分的理论研究和写作.

要特别感谢Js老师, 没有他的催更, 这篇文章不知道什么时候才会得到更新; 没有他的催更, 我将也没有机会进行如此令人兴奋的研究! {\footnotesize \color[rgb]{.7,.7,.7} 所以Js老师稿费结一下.}

此外, 到目前为止, 本文还有诸多不完善之处, 感谢大伙的批评指正!

\vspace{\stretch{2}}
\begin{flushright}
    中指君

    写于2025年10月12日
\end{flushright}

\vspace{\stretch{1}}
\pagebreak

\subsection{第一版感言}

这是本人第一次制作游戏攻略, 也应该是第一篇基于\LaTeX 写成的《14种扫雷变体》文档. 

在被《14种扫雷变体》的玩法和\LaTeX 出色的排版效果吸引后, 在今年年初, 我就产生了使用\LaTeX 创作《14种扫雷变体》相关攻略的想法. 为了满足在攻略中绘图的需要, 我制作了基于Ti\emph{k}Z的宏包Ti\emph{k}Z-14mv, 本文的所有图片都是使用以上两个宏包绘制的. 同时, 这应该是Ti\emph{k}Z-14mv第一次被使用在实际的攻略中.

此前, OQ格点定理常常在群里被提到, 但当时我并不了解其具体内容. 直到实际游玩\varible{O}\varible{Q}题板时, 有感其趣味性和OQ格点定理中蕴含的数学思维, 于是我顺理成章地决定将其写成攻略, 以便于更多的玩家了解. 我本计划写一个系统的\varible{O}\varible{Q}攻略, 并将本文作为其中的一部分. 但考虑到本人的水平有限, 恐怕无法撑起一篇完整的攻略, 因此仅仅写了OQ格点定理的部分.

此外, 格点定理不仅在出现在\varible{O}\varible{Q}题板中, 在其他规则中也有相应的版本, 例如QS、DQ格点定理等. 这同时说明 其中蕴含的数学思维是相通的. 我也有计划在本文中简要说明其他规则下的格点定理, 不过目前我还没有具体涉足过它们. 希望在本文的未来版本中, 能够加入其他规则下的格点定理.

在本文章的写作过程中, 我得到了很多群友的建议和指正, 特别要感谢灵秀之韵老师和sgo37老师, 他们分别贡献了\ \hyperref[inf]{\varible{O}\varible{Q}题板的下确界}和\hyperref[p2]{性质2}\ 的证明. 另外, 还要感谢群友们对本文章的支持和鼓励.

\vspace{\stretch{2}}
\begin{flushright}
    中指君

    写于2025年4月30日
\end{flushright}
\vspace{\stretch{1}}
\pagebreak

\appendix

\section{补充材料}
\subsection{\varible{O}\varible{Q}规则下非雷区域的边界为简单多边形}
群友sgo37给出了一个涉及图论的\hyperref[p2]{性质2}\ 的证明, 为了使得对图论了解程度不同的读者都能理解本证明, 先不那么系统地介绍一下图论的基本概念.

首先, 图是一些点(称为\textbf{顶点})和点之间的连线(称为\textbf{边})构成的, 在图论研究中,我们只关心顶点和边的关系, 而点和每个点的位置, 线的形状如何都不在考虑范围内.

不考虑一条边的两端是同一个顶点, 或者两个顶点间有多余一条边的情况(称为``简单的''), 与$V_i$关联(连接)的边数, 称作顶点$V_i$的度数, 记作$\deg(V_i)$, 由于一个边连接两点, 所以一个图的边数$E$是所有点度数之和的一半, 也即
$$
E = \frac{1}{2}\sum_{V}\deg(V_i).
$$

另外, 如果一个图能画成所有边都不相交的样子, 则称这个图为平面图, 有欧拉公式:
$$
V - E + F = 2.
$$
其中$V$为顶点数, $E$为边数, $F$为面数(可理解为平面图被分成区域的数量).

于是, 引出\hyperref[p2]{性质2}\ 的证明:
\begin{proof}
    \label{proof:SimplePolygon}
    首先, 非雷区域是无``洞''的, 换言之, 非雷区域是单连通区域. 否则雷区域与题板外部无法四方向联通, 这与\varible{O}规则矛盾. 因此, 非雷区域的边界是一条闭合曲线.
    那么, 将所有边界格点作为顶点, 相邻两边界格点于非雷区域边界上的连线作为边, 成为一个简单平面图, 有如下性质:
    \begin{enumerate}
        \item 其只有两个面, 对应非雷区域和雷区域及题板外部;
        \item 每个顶点的度数$\deg(V_i) = 2\ \text{或}\ 4$
    \end{enumerate}
    其边数$\displaystyle E = \frac{1}{2}\sum_{V}\deg(V_i)$, 面数为$F = 2$
    由平面图的欧拉公式
    $$
    V - E + F = 2
    $$
    得
    \begin{align*}
        V - \frac{1}{2}\sum_{V}\deg(V_i) + 2 &= 2 \\
            V - \frac{1}{2}\sum_{V}\deg(V_i) &= 0 \\
                           \sum_{V}\deg(V_i) &= 2V\\
                                   \deg(V_i) &= 2
    \end{align*}
    由此可知, 每个边界格点的度数均为2, 即边界不会自交, 即边界是一个简单多边形.
\end{proof}
这是一个\varible{O}题板的性质. 同样表明, \varible{O}题板中不含``扭断'', 因为扭断结构中中心格点的度数为4.

\subsection{\varible{O}\varible{Q} 题板雷数的下确界}
\label{inf}
此前证明了$R$的下界为$\left\lceil \frac{(m-1)(n-1)}{2} \right\rceil$, 要证明其同时为下确界,只需给出一种满足要求的构造方式即可. 如下的构造由群友灵秀之韵给出:
\begin{itemize}
    \item 当$m,n$均为偶数时, 取如\ \ref{even}\ 的分布;
    \item 当$m,n$其中一个为奇数时, 取如\ \ref{odd}\ 的分布.
\end{itemize}
\fmvsetup{
    tmargin = 0.2cm,
    lmargin = 0.2cm,
    rmargin = 0.2cm,
    bmargin = 0.2cm,
}
\begin{figure}[ht]
    \subfloat[$m,n$均为偶数]{
        \label{even}
        \begin{tikzpicture}
            \begin{fmv}{8}{10}
                \\
                & f & f & f & f & f & f & f & f & f \\
                \\
                & f & f & f & f & f & f & f & f & f \\
                \\
                & f &   & f & f & f &   & f & f & f  \\
                & f & f & f &   & f & f & f &   &    \\
                &   &   &   &   &   &   &   &   & f  \\
            \end{fmv}
        \end{tikzpicture}
    }
    \subfloat[$m,n$其中一个为奇数]{
        \label{odd}
        \begin{tikzpicture}
            \begin{fmv}{6}{11}
                \\
                & f & & f & & f & & f & & f \\
                & f & & f & & f & & f & & f \\
                & f & & f & & f & & f & & f \\
                & f & & f & & f & & f & & f \\
                & f & & f & & f & & f & & f \\
            \end{fmv}
        \end{tikzpicture}
    }
\end{figure}

\subsection{蛇面积方程的证明}
\label{proof:snack-area}
\todo{这个我不会证!}
\end{document}